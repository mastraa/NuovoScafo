\chapter{La stabilità}

La stabilità di un corpo descrive la sua attitudine a cambiare la sua configurazione quando perturbato in prossimità di un punto di equilibrio. In architettura navale è uno degli aspetti più importanti della fase di progettazione dell'imbarcazione.
In particolare si deve analizzare la stabilità longitudinale e quella trasversale, ciò che influenza la stabilità è la distribuzione dei pesi e la forma della carena.

Si parla di equilibrio \underline{positivo} se lo scafo perturbato (ad esempio da una folata di vento) tende a riportarsi nella posizione iniziale, \underline{indifferente} se lo scafo perturbato rimane nella sua posizione, \underline{negativo} se dopo la perturbazione tende a fuggire dalla posizione di partenza.

	\section{Obiettivo}
	Una barca a vela deve essere ovviamente stabile, ma questo non significa che debba sempre navigare in posizione 'dritta', ovviamente il caso dello skiff è un'eccezione, può anche trovarsi a navigare in posizione coricata e di conseguenza la forma dello scafo deve eventualmente prevedere questa opportunità.

	\section{Stabilità di peso}
	Come si evince dal nome la stabilità di peso riguarda la distribuzione dei pesi sulla barca, nel nostro caso, poichè più del $50\%$ del peso è composto dall'equipaggio e da attrezzatura mobile, lo studio dei pesi assume un'importanza minore.
	\'E ovviamente fondamentale calcolare il dislocamento, la linea di galleggiamento e quindi tutte le nostre linee d'acqua dipendono dalla sua posizione. Fatto questo però, nel nostro caso, è sufficiente fare in modo che i pesi siano il più possibile concentrati vicino al baricentro dello scafo per evitare che, a seguito di una forzante, si ingenerino delle coppie che portano la barca a rollare e beccheggiare sprecando così energia.
	
	\section{Stabilità di forma}
	Al contrario è fondamentale la scelta della forma dell'imbarcazione.



Differente è invece il caso della stabilità di forma in quanto possiamo andare a disegnare uno scafo più stabile o più 'ballerino' a seconda della forma che diamo alla parte istantaneamente immersa. Probabilmente è più facile spiegare questa situazione con un esempio: supponiamo che per un motivo ignoto (non interessa alla trattazione) la barca tenda a sbandare sul lato di dritta, in questo modo avremo che un volume X a dritta verrà immerso, provocando una spinta di galleggiamento, mentre un volume Y a sinistra emergerà.
A seguito di questa modifica della sezione immersa avremo una differente spinta idrostica (che potrebbe portare ad avere un minor pescaggio della scafo qualora il volume inizialmente immerso sia maggiore di quello emerso) ma soprattutto si generarà un momento dovuto allo spostamento a dritta del punto di applicazione della risultante delle spinte idrostiche. Lo sbandamento sarà tanto maggiore quanto più dobbiamo spostare tale punto.
Tanto più le sezioni della barca saranno squadrate (e larghe) tanto maggiore sarà lo spostamento del punto di applicazione con un piccolo sbandamento e tanto meno la barca soffrirà di rollio.

Questo principio influenza ovviamente anche il beccheggio, qui però entrano in gioco maggiormente anche le onde e le spinte che queste creano prevalentemente nella zona di prua essendo questa generalmente molto leggera: dislocare troppi volumi nella zona di prua (e quindi molto lontani dal centro di rotazione della barca che è di solito spostato a poppa rispetto al centro geometrico della barca) finirà per rendere la barca molto nervosa su mari agitati.