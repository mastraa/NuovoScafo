\chapter{La stabilità}
Per il tipo di imbarcazione che dobbiamo disegnare è bene fare una piccola disquisizione sulla stabilità, essa è influenzata da due parametri: il peso e la forma. Poichè il peso preponderante per le nostra imbarcazione (più del $50\%$) è rappresentato da corpi mobili (equipaggio, vele, boma) è difficile prevedere come questo influirà la navigazione e parte di questo lavoro è demandato alla sensibilità dell'equipaggio che con la sua posizione dovrà gestire l'assetto dello skiff.

Differente è invece il caso della stabilità di forma in quanto possiamo andare a disegnare uno scafo più stabile o più 'ballerino' a seconda della forma che diamo alla parte istantaneamente immersa. Probabilmente è più facile spiegare questa situazione con un esempio: supponiamo che per un motivo ignoto (non interessa alla trattazione) la barca tenda a sbandare sul lato di dritta, in questo modo avremo che un volume X a dritta verrà immerso, provocando una spinta di galleggiamento, mentre un volume Y a sinistra emergerà.
A seguito di questa modifica della sezione immersa avremo una differente spinta idrostica (che potrebbe portare ad avere un minor pescaggio della scafo qualora il volume inizialmente immerso sia maggiore di quello emerso) ma soprattutto si generarà un momento dovuto allo spostamento a dritta del punto di applicazione della risultante delle spinte idrostiche. Lo sbandamento sarà tanto maggiore quanto più dobbiamo spostare tale punto.
Tanto più le sezioni della barca saranno squadrate (e larghe) tanto maggiore sarà lo spostamento del punto di applicazione con un piccolo sbandamento e tanto meno la barca soffrirà di rollio.

Questo principio influenza ovviamente anche il beccheggio, qui però entrano in gioco maggiormente anche le onde e le spinte che queste creano prevalentemente nella zona di prua essendo questa generalmente molto leggera: dislocare troppi volumi nella zona di prua (e quindi molto lontani dal centro di rotazione della barca che è di solito spostato a poppa rispetto al centro geometrico della barca) finirà per rendere la barca molto nervosa su mari agitati.